\begin{summary}

\indent The current trend in industry standards for aviation technology is towards technologies with more fuel efficient and less noisy vehicles and power systems. One concept which has been used to much success in marine propulsion applications, and has been identified for future potential fuel burn savings for aviation is the "Boundary Layer Ingesting" (BLI) propulsion system.  This technology has been investigated at the theoretical level for aviation applications over the years by a few authors and has been the subject of extensive research in recent years in academia, industry, and government, due to the increased synergy of the concept with new vehicle designs such as the hybrid wing body.  

\indent The benefit of the BLI propulsion configuration comes from the basic fact that ingesting a portion of the aircraft upper surface tends to re-energize the low velocity boundary layer flow and thereby increase the propulsive efficiency of the system.  This, however, is counteracted by the fact that gas turbine component tend to operate with less efficiency and stability when subject to heavily distorted flow conditions.  The design challenge for BLI, then, is to maximize the amount of boundary layer which can be ingested while minimizing the negative impact of BLI on the gas turbine operation.  However, this task is made difficult due to the strong multi-disciplinary nature of the interactions between the engine and the airframe.

\indent  For the civil aviation engine designer, BLI poses a problem at the conceptual level because their are many new interaction effects where data must be filled into cycle analysis models, and such data may not be available in conceptual design.  This leaves the cycle analyst with a difficult task to quantify these effects and understand the impact of the uncertainty in the interactions on the choices made with regard to the engine cycle, the number of engines chosen, and other such conceptual level considerations.  Methods used to date have employed simple approaches for cycle analyses whereby the boundary layer is characterized using data from a single CFD solution or from closed form boundary layer approximations.  The losses are sometimes ignored, or are modeled parametrically with independent efficiency parameters in the cycle model.  Furthermore, cycle analysis methods to date have typically only employed a single design point methodology, thus ignoring the impacts of BLI at important design points like take-off, top of climb, and sea-level-static conditions and also ignoring the fact that engines often operate at different inlet conditions during flight causing a disparity in engine thrust.  Finally, conceptual level approaches typically ignore the impact of fan operability on the design trades made.

\indent The present thesis presents a conceptual level method for cycle analysis which employs multiple design conditions and multiple inlet conditions in the propulsion system sizing process.  The effects of BLI are modeled using a physics based approach, but are also modeled probabilistically so that uncertainty in the loss parameters can be mapped to the system level performance to determine the most important factors and design conditions for future development and experimentation.  A method for sizing the propulsion system in the presence of uncertainty is proposed which can provide a means for designing for additional thrust given that loss parameters may be higher than initially estimated on the real vehicle.  These methods will be tested on a hybrid wing body vehicle model with ultra hi-bypass geared turbofan engines ingesting a portion of the aircraft boundary layer.  An investigation of the effects of operability constraints on the propulsion system design space will also be tested.


\end{summary}