\chapter{A General Method for BLI Propulsion System Design, Integration, and Analysis}
\section{Methodology Development}
\subsection{MDP Cycle Analysis}
\subsection{Requirements for a BLI Sizing Method}
\subsection{Methodology and Justification}
\section{BLI Modeling Phase}
The history of the analysis of boundary layer ingesting engines, much as the history of any analysis, is littered with varying levels of modeling fidelity and assumptions with often quite disparate results INSERT REFERENCE HERE.  The complicated nature of the question -- viscous and turbulent airframe-propulsion interaction -- often lends itself to convenient simplications for the sake of expediency in order to draw some initial conclusion about potential net benefits and viable configurations.  Once such decisions have been made, engineers are free to move on to the difficult work of determining the validity of the assumptions and analysis via higher order toolsets such as modern Navier-Stokes codes and the like.  This is all very typical of a usual design process, in which conceptual design begins with some crude assumption and is refined by later analysis and optimization.  However, the point of this thesis is to try to get closer to a feasible answer -- at least for the basic propulsion system cycle design and sizing -- before the aerodynamicists embark on refining the assumptions that went into making that decision.  Furthermore, it is intended to guide the aerodynamicist and experimentalist in appropriately directing finite resources for their efforts in the most productive directions (correct flight conditions, configurations, initial geometry, etc), and providing sufficient data back to the propulsion engineer in an iterative process which eventually converges on a solution.  It is with this basic project in mind that research question 1 is formulated and stated simply as follows:
\vspace{25pt}
\vspace{5mm}
\fbox{
  \parbox{\textwidth}{
Research Question 1:  What are the minimum requirements for conceptual level modeling of a boundary layer ingesting cycle model in order to reasonably construct the architecture and cycle design space of a BLI propulsion system?
\vspace{5 mm}
  }
}
\vspace{5mm}

Note that the question asks for the "minimum" modeling requirements for conceptual design.  In a sense, this is asking "what can we get away with" or "what is good enough" at the conceptual level, since obviously the best case scenario is to build a complicated fluid dynamics model, allow it to run on an infinitely powerful computer, and come back with an answer.  Unfortunately, no such computer exists and even if it did, we'd have to ask about exactly which design we are modeling to start.

In attempting to answer this question, we first take on some components of the answer as being trivial; one needs a reasonable engine cycle model to begin with, as well as thermodynamic component models for the constituent machinery and ducting;  one also needs some approximation of the vehicle flow field at the points where it interacts with the engine and a total clean vehicle drag which translates to a thrust requirement.  These are the first few blocks of the "BLI modeling phase" and it is somewhat obvious that they must be known to complete any analysis of the system.  

The component of the question which is far more interesting, however, is in quantifying the interaction between the flow field and propulsion system and its impact on system performance.  These interactions can broadly be classified into 3 regimes: power balance (or thrust balance), turbomachinery performance and efficiency, and engine stability.  The first two have been looked at by almost every author on the subject, while the latter has been studied by some component designers, aerodynamic engineers, and a few others INSERT REFERENCE HERE.  Stability, though it is certainly a dominant concern among technologists, planners INSERT REFERENCE HERE, and experimentalists INSERT REFERENCE HERE, has tended to take a "back seat" at the cycle analysis level, in part because it is a difficult subject to analyze, but also because it is often assumed that modern aerodynamic methods will solve the problem after the fact INSERT REFERENCE HERE.  For this reason, we will begin the analysis by ignoring the stall margin question and returning to it later to analyze this dubious partitioning of the problem.

The analysis will also be separated into two different operational modes: engine on-design and off-design.  On-design is the analysis which typically sets the size of the engine, while off-design is any operational condition which deviates from those conditions, including variations in flight Mach number, altitude, or throttle setting ("part power").  On-design analysis essentially sets the level of thrust production and efficiency that the system is capable of providing, while the off-design analysis determines the variation of those quantities over a range of operating conditions.  The ultimate system fuel burn which results is actually a function of both on-design and off-design analysis, since the mission analysis which is typical requires performance over a wide variety of flight conditions.  It is thus necessary to consider both modes of operation, but we will begin, perhaps logically, with the subject of engine on-design analysis.

\subsection{On Design Analysis with BLI}

Engine on-design cycle analysis is the standard term for thermodynamic analysis which determines the size, mass flow (therefore thrust capability), and efficiency of an engine.  The engine design process usually begins with this type of analysis, however the final selection of the design is usually determined by off-design performance (to be discussed later) over the entire aircraft mission during the cycle selection process INSERT REFERENCE HERE.  This analysis is often carried out at a single design point in the simplest case, though it is difficult to often determine the proper design point where a significant portion of cycle choices at that point will yield feasible off-design performance.  Schutte INSERT REFERENCE HERE showed how modern cycle analysis tools and techniques can be used to implement a multi-design point approach (MDP), the purpose of which is to "ensure the feasbility of all cycle designs for a particular application".  This principle ought to -- and, in fact, does -- apply in the case of BLI, however the theoretical and academic work which implements BLI on-design analysis has largely ignored the issue of off-design performance and proceeds with determining performance differences at a nominal cruise point.  For BLI systems, especially those applied to the HWB aircraft concept, the typical flight condition assumed for the analysis is in excess of Mach 0.7, though typically closer to a 0.8-0.85 range.  The usual design point in these studies is a "top of climb" -- or maximum mass flow -- condition where corrected thrust required is largest.

The process for assessing the benefit of BLI at the design point chosen typically comes down to finding some means of translating the impact of the ingestion of the boundary layer on the propulsive efficiency.  The two equations shown below from Smith and Plas exhibit this mode of thinking about the problem.  Both equations represent the propulsive efficiency of a BLI ingesting propulsor in terms of some measure of the level of boundary layer ingested ( D/T for Smith, $\beta$ for Plas) and in terms of parameters describing the character of the wake or boundary layer being ingested.  The equations are slightly different since Smith and Plas are analyzing different conceptual configurations and propulsor designs and because of modeling assumptions.  However, the core objective remains the same:  to describe the benefit gained from BLI as a function of the amount of wake ingested and boundary layer parameters.  Fundamentally, the results from this analysis are similar:  ingesting more boundary layer provides a bigger benefit.  Other methods of analysis have been used, which involve translating the boundary layer velocity profile into an equivalent "ram drag" reduction -- sometimes called the "ram drag approach" -- and has yielded similar results.  None of these approaches are specific to engine on-design, as they could in principle be used to quantify off-design performance benefit as well, but they are typically applied at the engine design point in the sizing and cycle design analysis.
  
   \begin{equation}\eta_{prop} = \frac{2}{\displaystyle\frac{V_j}{V_o} + 1 - \frac{D}{T}\Big[\displaystyle\frac{V_j}{V_o} -1 + R (1-K)\Big]}  \label{Smith_Propulsive_Efficiency}\end{equation}%

   \begin{equation}\eta_{prop} = \frac{\Big(1+\beta\Big)}
					{\displaystyle \frac{H^*}{2} + \beta\Big[1+\frac{u_j-u_\infty}{2u_\infty}\Big]}\label{Plas_Propulsive_Efficiency}\end{equation}%

\vspace{15pt}
This standard observation from the BLI literature is therefore formulated as follows:

\vspace{1pt}
\vspace{5mm}
\fbox{
  \parbox{\textwidth}{
\textbf{Observation 1}:  The performance benefit of boundary layer ingestion systems is generally a function of the ratio of the ingested drag to the uningested drag -- or net thrust required.
\vspace{5 mm}
  }
}
\vspace{5mm}

The obvious question which arises from this observation pertains to which factors determine the percentage of BLI ingested in relation to the net thrust.  This question obviously relates to the manner in which one calculates the percentage of BLI ingested.  
Smith \cite{Smith1993} considers the case of a wake ingesting propeller and gives an equation for the amount of BLI ingested in terms of the momentum thickness as shown in equation \ref{Smith_Ingested_Drag}.
   \begin{equation}D = \rho V_o^2 \theta \label{Smith_Ingested_Drag}\end{equation}%
This equation is essentially a momentum balance approach which attempts to quantify the change in the net thrust of the vehicle due to wake ingestion as a function of the momentum deficit.  Here the density, velocity, and wake momentum deficit are defined at the entry of the propeller "actuator" disk.  This equation is the starting point for the development of equation \ref{Smith_Propulsive_Efficiency} and the D/T contained in that equation results from the form of equation \ref{Smith_Ingested_Drag}.  The ingested drag from \ref{Smith_Ingested_Drag} is seen as contingent upon the size of the momentum deficit in the wake.  Note that momentum deficit here is defined as a momentum area rather than a thickness as sometimes defined for 1-D boundary layers.  This means that this parameter is really a function of both the general momentum thickness of the wake and also the extent of the wake which is actually ingested.  The combined deficit integral over the control volume entrance then gives the final value of the wake momentum area deficit.  The point is that both the character of the wake and the amount of wake ingested has an influence on the amount of drag ingested.

Smith proceeds with a parametric approach with regard to the ratio of ingested drag to thrust and looks at the effect of this parameter on the propulsive efficiency as a function of the thrust loading coefficient $(C_{th})$.  He concludes his analysis by stating the following:  
\begin{quote}
 \textit{...for best efficiency the propulsor should be positioned and sized to ingest as much wake fluid as possible (increase D/T), but after that, making it still larger does not pay off in propulsive efficiency and would have other adverse effects such as increased weight.}
\end{quote}
This analysis assumes that at a certain point, the propulsor will be large enough to ingest the majority of the wake and therefore any attempt to make it larger would have diminishing returns due to weight increase.  For configurations such as those where the wake is distributed along a very large span such as that for the HWB aircraft, this analysis may break down and there may be additional gains from making the propulsors even larger or distributing a very large number of smaller propulsors across the upper surface.  In any case, the operative principle which is established here is that there is an important relationship between the sizing and positioning of the engine and the amount of boundary layer that can be ingested.  

Plas \cite{Plas2007} gives a slightly different expression for the calculation of the ingested drag percentage as show in equation \ref{Plas_Ingested_Drag}.

\begin{equation}D_w =  \rho u^2 \Big(\frac{u}{u_o}\Big)^{H_{avg}}\theta_o b\label{Plas_Ingested_Drag}\end{equation}%

In this case, the ingested drag is calculated based on a dissipation or energy based approach originally defined by Drela \cite{Drela2009}, where the ingested drag is essentially the dissipation coefficient divided by the free-stream velocity to convert it into the amount of equivalent thrust saved by the ingestion.  This equation is similar to equation \ref{Smith_Ingested_Drag} except that it is using the kinetic energy deficit rather than the momentum deficit though it calculates this by approximating the kinetic energy shape factor from the average shape factor.  

The variable "b" here is a term representing the "span" of the ingested boundary layer.  This variable essentially translates the 1-D kinetic energy thickness into an equivalent ingested area defect and therefore produces the correct thrust/drag modification.  The details of whether to use momentum or energy based approches can be discussed, but again the main point to be seen here is that benefit of wake ingestion ultimately relies upon some character of the boundary layer profile (momentum or KE deficit) and the extent or span of the ingested streamtube in relation to the overall required thrust of the system.  It is worth mentioning that there are many other studies essentially corroborating this basic principle and focusing on the relationship between the amount of BLI ingested and efficiency increase.  Observation 2 is therefore formulated as follows:

\vspace{1pt}
\vspace{5mm}
\fbox{
  \parbox{\textwidth}{
\textbf{Observation 2}:  The ratio of uningested drag to net thrust depends on both the value of the boundary layer defect properties and the span of the boundary layer ingested by the engine stream-tube.
\vspace{5 mm}
  }
}
\vspace{5mm}

\subsection{Off-Design Analysis}
\subsection{Hypothesis 1}

\subsection{Stall Margin and Stall Constraint}
\subsection{Hypothesis 2}

\section{Architecture Integration Phase}
\subsection{Integration Issues and Observations}
\subsection{Hypothesis 3}
\subsection{Design Possibilities}
\subsection{Hypothesis 4}

\section{Vehicle Matching Phase}
\subsection{General Criteria for BLI Sizing Condition}
\subsection{Hypothesis 5}
\subsection{Vehicle Matching Methodology}
\subsection{Integration with MDP Analysis}

